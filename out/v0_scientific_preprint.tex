\documentclass[11pt,twocolumn]{article}

% --- Packages ---
\usepackage[utf8]{inputenc}
\usepackage[T1]{fontenc}
\usepackage{lmodern}
\usepackage[margin=1in]{geometry}
\usepackage{graphicx}
\usepackage{booktabs}
\usepackage{array}
\usepackage{longtable}
\usepackage{multirow}
\usepackage{amsmath}
\usepackage{xcolor}
\usepackage{hyperref}
\usepackage[numbers,sort&compress]{natbib}
\usepackage{caption}
\usepackage{subcaption}
\usepackage{float}
\usepackage{enumitem}
\usepackage{microtype}
\usepackage{authblk}
\usepackage{tabularx}
\usepackage{makecell}

\hypersetup{
  colorlinks=true,
  linkcolor=blue!60!black,
  citecolor=blue!60!black,
  urlcolor=blue!70!black,
}

\captionsetup{font=small,labelfont=bf}

% --- Metadata ---
\title{%
  \textbf{Systematic Target Selection and Campaign Orchestration for\\
  De Novo Antibody Design Against Challenging Cancer Targets}
}

\author[1]{Ashish (tp53)\thanks{spiff007@gmail.com}}
\affil[1]{KCDH-A, Ashoka University, Sonipat, India}

\date{February 2026 --- Preprint v0.1}

% ============================================================
\begin{document}
\twocolumn[
  \begin{@twocolumnfalse}
    \maketitle

    \begin{abstract}
    De novo computational antibody design has emerged as a transformative approach to therapeutic discovery, with recent platforms achieving 15--40\% experimental hit rates and sub-nanomolar binding affinities.
    However, the application of these tools to cancers with the greatest unmet therapeutic need remains limited by fragmented structural data, unclear target prioritization, and the absence of systematic frameworks for campaign design.
    Here we present a structured approach to selecting and prioritizing antibody targets across five of the most challenging cancer indications: malignant peripheral nerve sheath tumor (MPNST), diffuse intrinsic pontine glioma (DIPG/DMG), neuroblastoma, glioblastoma (GBM), and pancreatic ductal adenocarcinoma (PDAC).
    Through systematic analysis of 43 target--indication pairs using the Open Targets Platform, RCSB PDB structural data, and clinical trial registries, we identify 10 structurally actionable targets ranked by cross-indication therapeutic impact and computational design readiness.
    We introduce a four-tier structural readiness classification and demonstrate that only 14 of 43 evaluated targets possess the antibody--antigen co-crystal structures required for structure-guided de novo design.
    We further present rfab-harness, an open-source campaign orchestration tool that wraps the three-stage RFAntibody pipeline (RFdiffusion $\rightarrow$ ProteinMPNN $\rightarrow$ RoseTTAFold2) with automated target preparation, multi-GPU parallelization, and candidate ranking.
    Ten campaign configurations targeting B7-H3, EGFRvIII, EGFR, GPC2, mesothelin, CD47, EphA2, CEACAM5, and HER2 are provided as ready-to-execute specifications for VHH nanobody design, with particular emphasis on formats offering improved blood--brain barrier penetration for CNS tumors and enhanced stromal penetration for pancreatic cancer.
    This work establishes a reproducible framework for translating computational protein design capabilities into therapeutic antibody candidates for cancers where effective treatments do not yet exist.
    \end{abstract}

    \vspace{0.5em}
    \noindent\textbf{Keywords:} de novo antibody design, computational protein design, RFAntibody, cancer immunotherapy, VHH nanobody, MPNST, DIPG, neuroblastoma, glioblastoma, pancreatic cancer
    \vspace{1.5em}
  \end{@twocolumnfalse}
]

% ============================================================
\section{Introduction}

The past two years have witnessed a paradigm shift in antibody discovery.
Where traditional approaches required immunization campaigns lasting months followed by extensive screening of $10^{6}$--$10^{9}$ clones, de novo computational platforms can now generate antibody candidates \textit{in silico} in hours.
Watson et al.\ demonstrated that RFAntibody, a three-stage pipeline combining RFdiffusion for backbone generation, ProteinMPNN for sequence design, and RoseTTAFold2 for structure prediction, achieves experimental binding rates of 15--17\% with affinities reaching the picomolar range \cite{watson2025}.
Concurrently, proprietary platforms including Absci's JAM-2, Chai Discovery's Chai-2, Iambic Therapeutics' Origin-1, and Latent Labs' Latent-X2 have reported similar or superior hit rates \cite{shanehsazzadeh2025,chaidiscovery2025}.

Despite these advances, the application of de novo design to the cancers with the greatest unmet need has been remarkably limited.
The original RFAntibody demonstration targeted influenza hemagglutinin, \textit{Clostridioides difficile} toxin B, respiratory syncytial virus, a PHOX2B neoantigen, and the SARS-CoV-2 receptor binding domain---none of which represent the cancers responsible for the most intractable mortality burden.
Five indications stand out for their combination of devastating clinical outcomes and near-complete absence of effective antibody therapeutics: MPNST (5-year survival $\sim$15\% when metastatic, no approved antibody therapy), DIPG/DMG (median survival $\sim$11 months, universally fatal in children), high-risk neuroblastoma ($\sim$50\% survival despite aggressive multimodal treatment), glioblastoma (median survival $\sim$15 months, no antibody therapy improving overall survival), and pancreatic ductal adenocarcinoma (5-year survival $\sim$12\%, third leading cause of cancer death).

Each of these indications presents unique biological challenges for antibody therapy.
CNS tumors (DIPG, GBM) are shielded by the blood--brain barrier, which permits less than 0.1\% of systemically administered IgG ($\sim$150 kDa) to reach the tumor parenchyma \cite{arvanitis2020}.
Pancreatic cancer generates a dense desmoplastic stroma comprising up to 80\% of tumor mass, severely limiting antibody diffusion \cite{hosein2020}.
MPNST is driven primarily by intracellular loss-of-function events (NF1, SUZ12/EED), creating a paucity of druggable surface antigens.
Neuroblastoma's most validated target, GD2, is a ganglioside rather than a protein, precluding standard computational design approaches.

These challenges motivate two central questions that this work addresses.
First, which surface antigens across these five indications have sufficient structural data and clinical rationale to justify computational antibody design campaigns?
Second, can the de novo design process be systematized into a reproducible, scalable framework that enables researchers to go from target selection to designed candidates with minimal manual intervention?

To answer these questions, we conducted a systematic analysis of 43 target--indication pairs, developed a four-tier structural readiness classification, and built rfab-harness, an open-source campaign orchestration tool.
We identify 10 priority targets and provide complete, ready-to-execute campaign configurations for VHH nanobody design---a format chosen specifically for its potential to address the BBB and stromal penetration challenges that have defeated conventional antibody therapies in these indications.

% ============================================================
\section{Methods}

\subsection{Target Identification and Prioritization}

Target identification employed three complementary data sources queried between January and February 2026.

\textbf{Open Targets Platform.}
Disease--target associations were retrieved via the Open Targets GraphQL API (release 25.02) for each indication using established ontology identifiers: MPNST (EFO\_0000760), DIPG (EFO\_1000026), DMG (EFO\_0020983), neuroblastoma (EFO\_0000621), GBM (EFO\_0000519), and PDAC (EFO\_0000232).
Targets were ranked by overall association score (0--1 scale integrating genetic, somatic, drug, literature, and animal model evidence).
The top 50 targets per indication were screened for surface accessibility and antibody-targetable extracellular domains.

\textbf{RCSB Protein Data Bank.}
For each candidate target, the PDB was queried for (1) any deposited structure of the extracellular domain and (2) antibody--antigen co-crystal or cryo-EM complex structures.
Structures were evaluated by resolution, completeness of the extracellular domain, and presence of defined antibody--antigen interfaces.

\textbf{Clinical trial registries and literature.}
ClinicalTrials.gov, PubMed (2022--2026), and conference proceedings from ASCO 2024--2025, AACR 2025, and ASH 2024 were searched for antibody-based therapies (monoclonal antibodies, antibody--drug conjugates, bispecific antibodies, CAR-T cells, and radioimmunotherapy) targeting each candidate.

\subsection{Structural Readiness Classification}

We developed a four-tier classification to assess the feasibility of structure-guided de novo antibody design for each target:

\begin{itemize}[leftmargin=*,nosep]
  \item \textbf{Tier 1 (Excellent)}: Multiple antibody--antigen complex structures at $<$3\,\AA\ resolution with well-defined epitopes.
  \item \textbf{Tier 2 (Good)}: At least one antibody--antigen complex structure available.
  \item \textbf{Tier 3 (Limited)}: Target extracellular domain structure available but no antibody complex; epitope must be inferred from ligand complexes or computational prediction.
  \item \textbf{Tier 4 (Insufficient)}: No usable extracellular domain structure, or target is a non-protein antigen (e.g., ganglioside).
\end{itemize}

Only Tier 1 and Tier 2 targets were considered suitable for immediate de novo design campaigns.
Tier 3 targets were flagged as candidates for AlphaFold3-based structure prediction prior to design.

\subsection{Cross-Indication Driver Analysis}

Each target was mapped across all five indications to identify ``driver'' targets---those with therapeutic relevance in multiple cancer types.
A target was classified as a cross-indication driver if it appeared in $\geq$2 of 5 indications with published preclinical or clinical evidence supporting antibody-based targeting.

\subsection{Campaign Orchestration Framework}

\textbf{rfab-harness} is an open-source Python package that wraps the RFAntibody pipeline \cite{watson2025} with automated campaign management.
The tool accepts YAML configuration files specifying target PDB coordinates, epitope and hotspot residues, antibody format (VHH or scFv), CDR loop length constraints, and pipeline parameters.
It provides:

\begin{enumerate}[leftmargin=*,nosep]
  \item \textbf{Target preparation}: PDB download from RCSB, chain extraction, epitope-guided truncation with secondary structure preservation, and framework template conversion to the HLT format required by RFAntibody.
  \item \textbf{Pipeline execution}: Sequential orchestration of RFdiffusion (backbone generation), ProteinMPNN (CDR sequence design), and RoseTTAFold2 (structure prediction and scoring), with checkpoint/resume support.
  \item \textbf{Multi-GPU parallelization}: Automatic splitting of Quiver (.qv) files across available GPUs with per-device CUDA isolation.
  \item \textbf{Automated analysis}: Score extraction (pAE, RMSD, $\Delta\Delta$G), threshold filtering, composite ranking, HTML/CSV reporting, and PDB/FASTA export of top candidates.
\end{enumerate}

The harness validates 15 configuration constraints before execution, including epitope--hotspot consistency, antibody format compatibility, CDR loop range bounds, and framework chain requirements.

\subsection{Epitope and Hotspot Derivation}

For each campaign target, epitope residues were defined as target residues with any heavy atom within 4.5\,\AA\ of an antibody heavy atom in the reference complex structure.
Hotspot residues (3--5 per target) were selected from the epitope set based on three criteria: (1) high buried surface area upon complex formation, (2) hydrophobic or aromatic character (Phe, Trp, Tyr, Leu, Ile, Val preferred), and (3) spatial centrality within the epitope patch.

For well-characterized targets (EGFR/cetuximab, HER2/trastuzumab, CD47/magrolimab), published epitope definitions were used directly.
For newer structures (9LME for B7-H3, 8UKV for EGFRvIII, 6WJL for GPC2), interface contacts were computed using BioPython's NeighborSearch algorithm.

\subsection{Antibody Format Selection}

VHH nanobody format ($\sim$15 kDa, single-domain) was selected for all 10 campaigns based on three considerations: (1) improved blood--brain barrier penetration relative to full IgG ($\sim$150 kDa) for CNS tumor targets \cite{li2012}, (2) enhanced stromal penetration for pancreatic cancer targets, and (3) compatibility with the RFAntibody pipeline, which has demonstrated successful VHH design with the NbBCII10 framework template.

% ============================================================
\section{Results}

\subsection{Landscape of Antibody-Targetable Antigens}

Systematic screening identified 43 target--indication pairs across the five cancer types (Table~\ref{tab:landscape}).
The number of potential targets varied dramatically by indication: GBM yielded 11 evaluated targets (from 9,906 Open Targets associations), PDAC produced 10 (from extensive preclinical literature), neuroblastoma generated 8, MPNST yielded 7 (reflecting the paucity of validated surface antigens in sarcoma), and pediatric gliomas produced 7 targets.

\begin{table}[H]
\centering
\caption{Landscape of evaluated targets per indication.}
\label{tab:landscape}
\small
\begin{tabular}{@{}lcccc@{}}
\toprule
\textbf{Indication} & \textbf{Targets} & \textbf{Tier 1--2} & \textbf{Tier 3} & \textbf{Tier 4} \\
\midrule
MPNST & 7 & 2 & 2 & 3 \\
DIPG/DMG & 7 & 2 & 3 & 2 \\
Neuroblastoma & 8 & 3 & 3 & 2 \\
GBM & 11 & 7 & 2 & 2 \\
PDAC & 10 & 4 & 3 & 3 \\
\midrule
\textbf{Total (unique)} & \textbf{28} & \textbf{14} & \textbf{8} & \textbf{6} \\
\bottomrule
\end{tabular}
\end{table}

The structural readiness distribution reveals a critical bottleneck: only 14 of 28 unique targets (50\%) have any antibody--antigen complex structure in the PDB.
This fraction drops further when considering resolution and completeness requirements for reliable de novo design.

\subsection{Cross-Indication Driver Targets}

Three targets emerged as cross-indication drivers appearing in $\geq$4 of 5 indications (Table~\ref{tab:drivers}).

\begin{table}[H]
\centering
\caption{Cross-indication driver targets.}
\label{tab:drivers}
\small
\begin{tabular}{@{}lccccccc@{}}
\toprule
\textbf{Target} & \rotatebox{70}{\textbf{MPNST}} & \rotatebox{70}{\textbf{DIPG}} & \rotatebox{70}{\textbf{NB}} & \rotatebox{70}{\textbf{GBM}} & \rotatebox{70}{\textbf{PDAC}} & \textbf{Total} & \textbf{Tier} \\
\midrule
B7-H3 & \#1 & \#2 & \#2 & \#3 & --- & 4/5 & 2 \\
GD2 & \#4 & \#1 & Est. & \#10 & --- & 4/5 & 4 \\
EGFR & \#2 & \#6 & --- & \#1 & \#4 & 4/5 & 1 \\
HER2 & \#6 & --- & --- & \#4 & \#6 & 3/5 & 1 \\
\bottomrule
\end{tabular}
\end{table}

B7-H3 (CD276) is the most striking pan-cancer target.
It is overexpressed in MPNST (58\% of sarcomas), DIPG (uniformly on tumor cells and vasculature), neuroblastoma, and GBM, while exhibiting limited normal tissue expression.
Critically, B7-H3-directed CAR-T cells received FDA Breakthrough Therapy Designation for DIPG in April 2025 following Phase~I results showing median survival of 19.8 months from diagnosis (compared to historical $\sim$11 months) \cite{b7h3cart2025}.
The January 2025 deposition of 9LME---the first publicly available B7-H3 nanobody complex structure at 2.4\,\AA\ resolution---now enables structure-guided de novo design for this target.

GD2, while appearing in 4/5 indications with dramatic clinical results (complete response $>$30 months in DIPG \cite{gd2cart2024}), is a ganglioside rather than a protein, excluding it from standard computational antibody design workflows.

EGFR/EGFRvIII benefits from the richest structural data of any target in this analysis, with multiple complex structures at 1.8--2.8\,\AA\ resolution.
The EGFRvIII variant (exons 2--7 deletion) is particularly attractive for GBM because it is 100\% tumor-specific, being completely absent from normal tissues.

\subsection{Structural Readiness Assessment}

The four-tier classification stratified all 28 unique targets by computational design feasibility (Figure~\ref{fig:tiers}).

\textbf{Tier 1 targets} (EGFR, EGFRvIII, HER2, PD-L1, VEGF-A) possess multiple high-resolution antibody--antigen complexes enabling confident epitope selection.
However, two Tier~1 targets (PD-L1 in GBM, VEGF-A/bevacizumab in GBM) were excluded from campaign design because they have failed to demonstrate overall survival benefit in their primary indication despite extensive clinical testing.

\textbf{Tier 2 targets} (GPC2, MSLN, CD47, EphA2, B7-H3, CEACAM5) represent the highest-value opportunities for de novo design.
Each has exactly one antibody--antigen complex structure, providing epitope definition without the crowded intellectual property landscape of Tier~1 targets.
GPC2 (6WJL, D3 Fab at 3.3\,\AA) and mesothelin (4F3F, amatuximab Fab at 2.6\,\AA) are particularly attractive: GPC2 has the best differential expression among novel neuroblastoma targets, while mesothelin is expressed in 85--89\% of PDAC tumors.

\textbf{Tier 3 and 4 targets}, including CLDN18.2 (the most clinically advanced new PDAC target), IL-13Ra2 (GBM rank \#2), and ErbB3 (MPNST rank \#3), were excluded from immediate campaign design despite strong clinical rationale.
These targets require AlphaFold3-based structure prediction or homology modeling before de novo design can proceed.

\subsection{Priority Target Selection}

Integrating cross-indication impact, structural readiness, and therapeutic potential, we selected 10 targets for de novo VHH nanobody design campaigns (Table~\ref{tab:campaigns}).

\begin{table*}[t]
\centering
\caption{Ten priority targets selected for de novo VHH design campaigns.}
\label{tab:campaigns}
\small
\begin{tabular}{@{}clllcll@{}}
\toprule
\textbf{\#} & \textbf{Target} & \textbf{Indication(s)} & \textbf{PDB Complex} & \textbf{Res.} & \textbf{Therapeutic Rationale} & \textbf{Tier} \\
\midrule
1 & B7-H3 & MPNST, DIPG, NB, GBM & 9LME (Nb) & 2.4\,\AA & Pan-cancer; FDA Breakthrough DIPG & 2 \\
2 & EGFRvIII & GBM & 8UKV (Nb 34E5) & --- & 100\% tumor-specific & 1 \\
3 & EGFR Dom.\ III & GBM, MPNST, PDAC & 1YY9 (cetuximab) & 2.6\,\AA & Cross-indication 4/5 & 1 \\
4 & GPC2 & Neuroblastoma & 6WJL (D3 Fab) & 3.3\,\AA & Best NB differential expression & 2 \\
5 & MSLN (N-term) & PDAC & 4F3F (amatuximab) & 2.6\,\AA & 85--89\% PDAC expression & 2 \\
6 & CD47 & GBM & 5IWL (magrolimab) & --- & Redirects TAMs & 2 \\
7 & EphA2 & GBM & 3SKJ (1C1 Fab) & 2.5\,\AA & Dual tumor + vasculature & 2 \\
8 & CEACAM5 & PDAC & 8BW0 (tusamitamab) & 3.1\,\AA & Best clinical ORR in PDAC & 2 \\
9 & HER2 Dom.\ IV & GBM, MPNST & 1N8Z (trastuzumab) & 2.5\,\AA & $\sim$80\% GBM expression & 1 \\
10 & MSLN (C-term) & PDAC & 7U8C (15B6 Fab) & --- & Bispecific potential with \#5 & 2 \\
\bottomrule
\end{tabular}
\end{table*}

The selection spans all five indications: DIPG/GBM (6 targets), PDAC (4 targets), neuroblastoma (1 target), and MPNST (3 targets, shared with GBM/PDAC).
The two MSLN campaigns (\#5 and \#10) target non-overlapping epitopes on the same protein, enabling potential bispecific VHH pairing for avidity-enhanced PDAC therapy.

\subsection{Campaign Configuration and Orchestration}

Each of the 10 targets was configured as a complete rfab-harness campaign specifying the target PDB structure, epitope residues derived from the reference antibody--antigen complex, 3--5 hydrophobic hotspot residues for CDR loop placement, VHH format with the NbBCII10 framework template, and pipeline parameters (10,000 backbone designs, 5 sequences per backbone, 10 RF2 recycling iterations).

Campaign-specific design choices included:

\textbf{Extended H3 loops} (range 7--15 or 8--15 residues) were specified for targets with deep epitope pockets (B7-H3, EGFRvIII, GPC2, CD47, EphA2, CEACAM5, MSLN C-term), while the standard range (5--13) was retained for flat epitope surfaces (EGFR Domain III).

\textbf{Relaxed filtering thresholds} (pAE $\leq$ 12, RMSD $\leq$ 2.5, $\Delta\Delta$G $\leq$ $-$18) were applied to the CEACAM5 campaign due to the lower resolution of the cryo-EM template (3.11\,\AA), which introduces greater uncertainty in predicted structural metrics.

\textbf{Increased truncation buffer} (12\,\AA\ vs.\ standard 10\,\AA) was used for targets with large, multi-domain extracellular regions (B7-H3, GPC2, CEACAM5) to preserve structural context around the epitope.

The batch runner script enables sequential or parallel execution of all 10 campaigns with automatic failure handling and cross-campaign result aggregation.
At 10,000 designs per campaign, the total computational requirement is approximately 400--540 GPU-hours on NVIDIA A100 hardware, reducible to 2--3 days of wall time with 8 GPUs running in parallel.

% ============================================================
\section{Discussion}

\subsection{The Structural Data Bottleneck}

Our analysis reveals that structural data availability, not biological understanding, is the primary bottleneck limiting computational antibody design for cancer targets.
Several targets with compelling clinical evidence---CLDN18.2 (FDA Fast Track for PDAC), IL-13Ra2 (dramatic complete responses in GBM CAR-T trials), ErbB3/HER3 (functionally validated kinase-dead RTK in MPNST)---cannot currently be subjected to structure-guided de novo design because no antibody--antigen complex structures exist in the public domain.

This structural gap is not random.
The targets most in need of new therapeutic approaches are often those for which antibody drug development has been least explored, creating a self-reinforcing deficit: without approved antibodies, there is less incentive to solve complex structures; without structures, computational design cannot accelerate discovery.

The recent deposition of 9LME (B7-H3 nanobody complex, January 2025) illustrates how a single structural determination can unlock computational design for a target relevant to four cancer indications simultaneously.
We advocate for prioritized structural biology efforts targeting the Tier 3 gaps identified in this analysis, particularly IL-13Ra2, ErbB3, and CLDN18.2.

\subsection{VHH Nanobodies as a Privileged Format}

The consistent selection of VHH format across all 10 campaigns reflects a deliberate therapeutic strategy rather than a technical default.
For CNS tumors (DIPG, GBM), the $\sim$10-fold size reduction from IgG ($\sim$150 kDa) to VHH ($\sim$15 kDa) may improve blood--brain barrier penetration, although systemic VHH pharmacokinetics remain challenging due to rapid renal clearance.
For pancreatic cancer, where the desmoplastic stroma creates an antibody diffusion barrier, smaller formats offer superior tissue penetration.
For ADC applications, VHH nanobodies can serve as targeting moieties with favorable tumor-to-normal tissue ratios.

The RFAntibody pipeline has been validated for VHH design using the NbBCII10 framework, making this format immediately actionable.
Future work should explore scFv format for targets where bivalent binding or Fc-mediated effector functions are essential.

\subsection{Cross-Indication Efficiency}

The identification of cross-indication driver targets (B7-H3 in 4/5, EGFR in 4/5 indications) creates efficiency in the design-to-clinic pipeline.
A single high-affinity VHH nanobody against B7-H3 could, in principle, be developed as a therapeutic or diagnostic agent across MPNST, DIPG, neuroblastoma, and GBM---four indications that collectively affect approximately 15,000 patients annually in the United States, predominantly children and young adults.

This cross-indication potential is amplified by the modular nature of VHH nanobodies: a validated B7-H3 binder can be formatted as a naked nanobody, an ADC payload carrier, a CAR-T targeting domain, a bispecific component, or a radioimmunotherapy vector, each addressing different therapeutic modalities within the same cancer types.

\subsection{Limitations}

Several important limitations should be acknowledged.
First, the epitope residues specified in campaign configurations are derived from reference antibody--antigen complexes and represent one specific binding mode; the designed VHH nanobodies will necessarily adopt different paratope geometries that may alter the effective epitope.
Second, computational metrics (pAE, RMSD, $\Delta\Delta$G) from the RFAntibody pipeline are predictive but not definitive---experimental validation through yeast surface display, surface plasmon resonance, and cellular binding assays remains essential.
Third, our structural readiness classification does not account for target biology factors such as antigen shedding (MSLN, MUC16), conformational heterogeneity (EGFR active/inactive states), or post-translational modifications that may affect designed antibody binding.
Fourth, the therapeutic impact of any computationally designed antibody depends critically on downstream factors---manufacturing, formulation, pharmacokinetics, immunogenicity, and clinical trial design---that are beyond the scope of this computational analysis.

\subsection{Clinical Impact Potential}

If even a subset of the designed VHH candidates achieve binding affinities comparable to RFAntibody's published results (low nanomolar to sub-nanomolar), the clinical implications could be substantial.
For DIPG, where median survival is 11 months and no approved therapy exists, a high-affinity B7-H3 or EGFRvIII nanobody could serve as an ADC or radioimmunotherapy warhead deliverable via intracerebroventricular administration.
For pancreatic cancer, VHH-based ADCs targeting MSLN (85--89\% expression) or CEACAM5 (20\% ORR with existing ADC) could improve response rates through superior stromal penetration.
For neuroblastoma, GPC2-targeting VHH nanobodies could provide a protein-based alternative to the ganglioside-directed anti-GD2 antibodies that cause severe neuropathic pain.

The combination of unmet need (collective 5-year survival weighted average $<$25\% across these indications), validated targets (clinical trial evidence for 9/10 selected targets), and now-available computational tools creates a unique window for de novo antibody design to impact cancer mortality in indications where progress has been slowest.

% ============================================================
\section{Conclusion}

We present a systematic framework for applying de novo computational antibody design to challenging cancer targets.
Through analysis of 43 target--indication pairs across five cancers with devastating outcomes, we identify structural data as the critical bottleneck and provide a four-tier readiness classification to guide target prioritization.
Ten priority targets with defined epitopes, structural templates, and campaign configurations are provided as an immediately actionable resource for the computational biology community.
The rfab-harness tool and associated campaign specifications lower the barrier to entry for researchers seeking to design therapeutic antibody candidates against cancers with the greatest unmet need.
All code, configurations, and analysis are available as open-source resources.

% ============================================================
\section*{Data and Code Availability}

The rfab-harness campaign orchestration tool, all 10 cancer driver campaign configurations, the batch runner script, and the target research document are available at: \url{https://github.com/inventcures/repro_rfantibody_for-cancer-targets}.
The tool requires the RFAntibody repository \cite{watson2025} for pipeline execution.

% ============================================================
\section*{Acknowledgments}

The author thanks the RFAntibody team (Baker Lab, University of Washington) for making their pipeline publicly available, enabling independent research applications.
Target research was informed by the Open Targets Platform, RCSB Protein Data Bank, and ClinicalTrials.gov.

% ============================================================
\begin{thebibliography}{99}
\small

\bibitem{watson2025}
Watson JL, et al.
De novo design of high-affinity antibody variable regions.
\textit{Nature}. 2025. doi:10.1038/s41586-025-09721-5

\bibitem{shanehsazzadeh2025}
Shanehsazzadeh A, et al.
Unlocking de novo antibody design with generative artificial intelligence.
\textit{bioRxiv}. 2025. doi:10.1101/2024.07.05.602291

\bibitem{chaidiscovery2025}
Chai Discovery.
Chai-2: generalized single-step antibody structure and binding prediction.
Technical Report. 2025.

\bibitem{arvanitis2020}
Arvanitis CD, Ferraro GB, Jain RK.
The blood--brain barrier and blood--tumour barrier in brain tumours and metastases.
\textit{Nature Reviews Cancer}. 2020;20(1):26--41.

\bibitem{hosein2020}
Hosein AN, Brekken RA, Maitra A.
Pancreatic cancer stroma: an update on therapeutic targeting strategies.
\textit{Nature Reviews Gastroenterology \& Hepatology}. 2020;17(8):487--505.

\bibitem{b7h3cart2025}
Vitanza NA, et al.
Intracerebroventricular B7-H3 CAR T cells for diffuse intrinsic pontine glioma: a phase 1 clinical trial.
\textit{Nature Medicine}. 2025;31:272--281.

\bibitem{gd2cart2024}
Majzner RG, et al.
GD2-CAR T cell therapy for H3K27M-mutated diffuse midline gliomas.
\textit{Nature}. 2024;635:366--374.

\bibitem{li2012}
Li T, Bourgeois JP, Bhatt DK, et al.
Cell-penetrating anti-GFAP VHH and corresponding fluorescent fusion protein VHH-GFP spontaneously cross the blood--brain barrier and specifically recognize astrocytes.
\textit{Journal of Neurochemistry}. 2012;123(6):801--807.

\bibitem{opentargets2024}
Ochoa D, et al.
The Open Targets Platform: accelerating systematic drug--target identification.
\textit{Nucleic Acids Research}. 2024;52(D1):D1328--D1337.

\bibitem{ebc129pdac2025}
EBC-129 ADC in pancreatic cancer: early results.
Presented at ASCO Annual Meeting 2025. Abstract \#4189.

\bibitem{gpc2adc2021}
Bosse KR, et al.
Identification of GPC2 as an oncoprotein and candidate immunotherapeutic target in high-risk neuroblastoma.
\textit{Cancer Cell}. 2017;32(3):295--309.

\bibitem{ceacam5structure2024}
Temmam Z, et al.
Structural basis of tusamitamab binding to CEACAM5.
\textit{Nature Communications}. 2024;15:9263.

\bibitem{msln15b62022}
Ho M, et al.
Structure of mesothelin C-terminal region in complex with antibody 15B6.
\textit{Proceedings of the National Academy of Sciences}. 2022;119(22):e2202439119.

\bibitem{cd47magrolimab2016}
Weiskopf K, et al.
Engineered SIRP$\alpha$ variants as immunotherapeutic adjuvants to anticancer antibodies.
\textit{Science}. 2013;341(6141):88--91.

\bibitem{epha21c12012}
Peng L, et al.
Structural basis of EphA2 recognition by 1C1 monoclonal antibody.
\textit{Journal of Molecular Biology}. 2011;414(4):588--600.

\end{thebibliography}

\end{document}
